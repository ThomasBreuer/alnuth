
\Chapter{Installation}

This package provides an interface between GAP 4 and KANT respectively
KASH, the shell of the computational algebraic number theory system KANT. By
now the interface can only be used on a Linux system.  KASH itself is not part
of this package.  It has to be obtained and installed independently of this
package. Alnuth works with KASH version 2.4 or 2.5.

%%%%%%%%%%%%%%%%%%%%%%%%%%%%%%%%%%%%%%%%%%%%%%%%%%%%%%%%%%%%%%%%%%%%%%%%%%%%%
\Section{Getting and installing KASH}
 
KASH is available at
\begintt 
            www.math.tu-berlin.de/~kant/download.html
\endtt
 
Note that you have to download two files for a complete installation
of KASH. For the installation of version 2.5 of KASH on a Linux
system you would do the following steps:
\beginlist
\item{1.} Download the files
    kash_2.5.6.common.tar.gz and kash_2.5.7.linux.tar.gz
    into the same directory on your system.
 
\item{2.} Unpack the files using tar.  This will create a directory
    KASH_2.5 containing among other files the KASH executable called
    kash.

\endlist

The place where KASH is located in your system is independent of the place
 where the Alnuth-package is installed.

%%%%%%%%%%%%%%%%%%%%%%%%%%%%%%%%%%%%%%%%%%%%%%%%%%%%%%%%%%%%%%%%%%%%%%%%%%
\Section{Installing this package}
 
This package is available at
\begintt 
           www.icm.tu-bs.de/ag_algebra/software/assmann/Alnuth
\endtt 

in form of a gzipped tar-archive or as an uncompressed tar-archive.
 
There are two ways of installing the package.  If you have permission
to add files to the installation of GAP 4 on your system you may
install the Alnuth-package into the pkg subdirectory of the GAP
installation tree.  If you do not have the permission to do that you
may install the Alnuth-package in your private area. In the latter case you
need to have a directory named pkg in your private area (for details see
the Section "ref:Installing a GAP Package" in the reference manual). 

Now move the alnuth.tar.gz or alnuth.tar file into the directory pkg and
unpack it:

\beginexample
    bash> tar xfz alnuth.tar.gz        # for the gzipped tar-archive
    bash> tar xf alnuth.tar         # for the uncompressed tar-archive
\endexample

%%%%%%%%%%%%%%%%%%%%%%%%%%%%%%%%%%%%%%%%%%%%%%%%%%%%%%%%%%%%%%%%%%%%%%%%%%%%%
\Section{Adjust the path of the executable for KASH}

This package needs to know where the executable for KASH is. 
In the default setting Alnuth will check if there is an executable
called `kash' in your search path (The search path is the 
set of directories through which your shell looks for executable 
programs when no absolute path is given. Type
`echo \$PATH'
in your terminal to see which directories are contained in your
search path.).

In Section "How to write a shell script which executes KASH"
we explain how to write a very short shell script,
which executes KASH, 
and indicate how you can assure that Alnuth finds it in the default
setting. 
We recommend to use this default setting.
An advantage of this method is, that in case 
of an installation of a new version of Alnuth you do not have to 
adjust the path to the executable of KASH again. 

If you do not want to use the default setting, then there are
two other possibilities.
 
If you are able to edit the file `pkg/alnuth/defs.g', then 
you can change the line 
\beginexample
    BindGlobal( "KANTEXEC", Filename( DirectoriesSystemPrograms( ), "kash" ) );
\endexample
to something like
\beginexample
    BindGlobal( "KANTEXEC", "mykash/kash -l mykash/lib" );
\endexample 
where `mykash' needs to be replaced with the directory where KASH
was installed. For example `mykash' could be replaced by
`/usr/local/KASH_2.5'. Please note that in case of a new installation 
of Alnuth you will have to edit the file `pkg/alnuth/defs.g'
again. Alternatively you can also change your personal `gaprc' file (see 
the Subsection "ref:The gaprc file" in the reference manual) for setting the
variable KANTEXEC to a proper value. To do this add the command line mentioned
above to `gaprc'.
 
The third possibility is to change the path to the executable within GAP using
one of the following two functions. To do this you first have to load the
package (see Section "Loading and testing the package").
\> SetKantExecutable( <path> )

adjusts the global variable `KANTEXEC' for the current GAP session. Depending
on your installation of KASH the string <path> has to be either the command
to start KASH in a terminal (for example `kash') 
or the complete path to the executable of KASH 
(for example `/usr/local/KASH_2.5/kash'). In
the latter case the library-path does not have to be specified, but is added
automatically. Thereby the message
\beginexample
    kash: hmm, I cannot find 'lib/init.g', maybe use option '-l <libname>'?
\endexample
will appear on the screen, which can be ignored.

To use
\> SetKantExecutablePermanently( <path> )

you need to be allowed to overwrite the file `pkg/alnuth/defs.g'. The function
does the same as `SetKantExecutable' and changes the file `pkg/alnuth/defs.g'
respectively in addition. Thus the value of the global variable `KANTEXEC' is
changed permanently. In case of a new installation of Alnuth,
you will have to run this command again. 

Both functions run a test whether <path> is a valid string for a filename of an
executable for KASH version 2.4 or 2.5.

If you want to set the path to the executable of KASH using the function
`SetKantExecutable' every time you start GAP, you could add
the command line `SetKantExecutable( <path> )' to your 
personal `gaprc' file (see the Subsection "ref:The gaprc file" in the 
reference manual). 

%%%%%%%%%%%%%%%%%%%%%%%%%%%%%%%%%%%%%%%%%%%%%%%%%%%%%%%%%%%%%%%%%%%%%%%%%%%%%
\Section{How to write a shell script which executes KASH}

In this section we explain how to write a shell script which executes
KASH. Such a script is needed if you want to use the default setting
of Alnuth for the execution of KASH 
(see Section "Adjust the path of the executable for KASH").

Switch to your home directory (`cd ~') and check (using `ls') if there
is a directory called `bin'. If not then create one with
`mkdir bin'. Change to the `bin' directory (`cd bin') and 
and open an empty file called `kash' with an editor of your choice.
Add the lines
\beginexample
     #!/bin/sh
     mykash/kash -l mykash/lib
\endexample
where `mykash' needs to be replaced with the directory where KASH was
installed. After this your file could look for example like this:
\beginexample
     #!/bin/sh
     /usr/local/KASH_2.5/kash -l /usr/local/KASH_2.5/lib
\endexample
Save  the file and close the editor. 
Then type `chmod u+x kash' in your terminal to make the script executable.

Now we have to assure that the directory `bin' is in your search path. 
Type `echo \$PATH' to check if this is the case. If not then you can add 
`bin' to the search path by typing
\beginexample
    bash> export PATH=$PATH:/home/user/bin
\endexample
where `/home/user' needs to be replaced by your home directory. 
If you want to extend your search path permanently you can 
add this line to the `.bashrc' file in your home directory. 

If you use a c-shell instead of bash, then you have to extend the 
search path with the command 
\beginexample
     set PATH = ( $PATH /home/user/bin )
\endexample
and edit the file `.cshrc'. 

%%%%%%%%%%%%%%%%%%%%%%%%%%%%%%%%%%%%%%%%%%%%%%%%%%%%%%%%%%%%%%%%%%%%%%%%%%%%%
\Section{Loading and testing the package}

To use this package you have to request it explicitly. This is done by calling

\beginexample
    gap> LoadingPackage("alnuth");
    Loading Alnuth 2.2.2 ...
    true
    gap>
\endexample
Once the package is loaded, it is possible to check the correct
    installation by running the test suite of the package with the command

\beginexample
    gap> ReadPackage( "alnuth", "tst/testall.g" );
\endexample
   




