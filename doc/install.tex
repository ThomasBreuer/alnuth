
\Chapter{Installation}

This package provides an interface between GAP 4 and KANT respectively
KASH, the shell
of the computational algebraic number theory system KANT.  KASH itself
is not part of this package.  It has to be obtained and installed
independently of this package.

%%%%%%%%%%%%%%%%%%%%%%%%%%%%%%%%%%%%%%%%%%%%%%%%%%%%%%%%%%%%%%%%%%%%%%%%%%%%%
\Section{Getting and installing KASH}
 
KASH is available at
\smallskip 
          \centerline{\tt www.math.tu-berlin.de/$\sim$kant/download.html}
\smallskip 
Note that you have to download two files for a complete installation
of KASH.  For the installation of version 2.4 of KASH on a Linux
system you would do the following steps:
{\parindent=25pt
\item{1.} Download the files
    kash_2.4.common.tar.gz and kash_2.4.1.linux.tar.gz
    into the same directory on your system.
 
\item{2.} Unpack the files using tar.  This will create a directory
    KASH_2.4 containing among other files the KASH executable called
    kash.  The place where KASH is put is independent of the place
    where the Alnuth-package is installed.
    \smallskip
}


%%%%%%%%%%%%%%%%%%%%%%%%%%%%%%%%%%%%%%%%%%%%%%%%%%%%%%%%%%%%%%%%%%%%%%%%%%
\Section{Installing this package}
 
This package is available at
\smallskip 
   \centerline{\tt http://cayley.math.nat.tu-bs.de/software/assmann/Alnuth}
 
in form of a gzipped tar-archive or as an uncompressed tar-archive.
 
There are two ways of installing the package.  If you have permission
to add files to the installation of GAP 4 on your system you may
install the Alnuth-package into the pkg subdirectory of the GAP
installation tree.  If you do not have the permission to do that you
may install the Alnuth-package in your private area.

\medskip
\leftline{\bf 4.2.1 Installation in the GAP 4 pkg subdirectory on a Unix system.}
 
\smallskip
    We assume that the archive file alnuth.tar.gz or alnuth.tar is
    present in pkg and that the current directory is pkg.  All that needs
    to be done is to unpack the archive.
 
\beginexample
    bash> tar xfz alnuth.tar.gz        # for the gzipped tar-archive
    bash> tar xf alnuth.tar         # for the uncompressed tar-archive
    bash> gap4
    [... startup messages ...]
    gap> RequirePackage("alnuth");
    true
    gap>
\endexample

 Now edit the file alnuth/read.g and change the line
 
\beginexample
    KANTEXEC := "kash"
\endexample
    to something like
\beginexample
    KANTEXEC := "mykash/kash -l mykash/lib"
\endexample
    where {\tt mykash} needs to be replaced with the directory where kash
    was installed.


\medskip
\leftline{\bf  4.2.2. Installation in a private directory}
 
    We assume that we are in a directory called  {\tt mygap}  which also
    contains the archive file of the Alnuth-package.
 
\beginexample
    bash> mkdir pkg
    bash> mv alnuth.tar.gz pkg
    bash> tar xfz alnuth.tar.gz
\endexample

    Now edit the file alnuth/read.g and change the line
\beginexample 
    KANTEXEC := "kash"
\endexample 
    to something like
\beginexample
    KANTEXEC := "mykash/kash -l mykash/lib"
\endexample 
    where {\tt mykash} needs to be replaced with the directory where kash
    was installed.
 
    When you start GAP 4 you have to use the option -l in the
    following manner:
 
\beginexample
    gap4 -l ";mygap"
\endexample 
    Note the semicolon!  It is important to have it there.  The effect
    is that the directory {\tt mygap} is appended to the list of
    directories which GAP searches for input data.  Note that you do
    not say {\tt mygap/pkg}.


\medskip
\leftline{\bf  4.2.3. Running the test suite}
 
    Once the package is installed, it is possible to check the correct
    installation by running the test suite of the package. 

\beginexample
    gap> Read( "mygap/pkg/alnuth/tst/testall.g" );
\endexample
    where {\tt mygap} needs to be replaced with the directory where gap
    was installed.









